\documentclass[fleqn,12pt]{article}
\usepackage[utf8]{inputenc}
\usepackage[margin=1.0in]{geometry}
\usepackage{amsmath}
\usepackage[per-mode=repeated-symbol]{siunitx}
\usepackage[nodisplayskipstretch]{setspace}


\begin{document}
\setlength{\abovedisplayskip}{2pt}
\setlength{\belowdisplayskip}{2pt}
\section*{\LARGE\underline{Physics Quiz 3 Formulas}}
$ \mu_0 = 4\pi \cdot 10^{-7} \si{\newton\per\ampere^2}$ , \, 
$\epsilon_0 = 8.854 \cdot 10^{-12} \si{\coulomb^2\per\newton\meter^2}$
\section*{Maxwell's Equations}
\begin{equation} \label{eq1}
 \mbox{\textbf{Gauss' Law for Electricity: }}\Phi_{E} = 
    \oint_A \vec{E} \cdot \vec{dA} \, = \oint_V ( \nabla \cdot \vec{E}) \cdot \vec{dV}
= \frac{q_\mathrm{enc}}{\epsilon_0}
\end{equation}

\begin{equation} \label{eq1}
 \mbox{\textbf{Gauss' Law for Magnetism: }}\Phi_{B} = 
    \oint_A \vec{B} \cdot \vec{dA} \, = \oint_V ( \nabla \cdot \vec{B}) \cdot \vec{dV}
= 0
\end{equation}

\begin{equation} \label{eq1}
 \mbox{\textbf{Faraday's Law: }}
    \oint_L \vec{E} \cdot \vec{dl} \, = \oint_A ( \nabla \times \vec{E}) \cdot \vec{dA}
= - \frac{d\Phi_{B}}{dt}
\end{equation}

\begin{equation} \label{eq1}
 \mbox{\textbf{Ampere's Law: }}
    \oint_L \vec{B} \cdot \vec{dl} \, = \oint_A ( \nabla \times \vec{B}) \cdot \vec{dA}
= \mu_{0}I_\mathrm{int} + \mu_{0}\epsilon_{0}\kappa\frac{d\Phi_{E}}{dt}
\end{equation}

\section*{Chapter 29}

\begin{equation} \label{eq1}
 \mbox{\textbf{Inductance: }}\mathcal{E}_\mathrm{ind} = -L \frac{dI}{dt}
 \quad \quad \quad \mbox{\textbf{Solenoid: }}L = \frac{\mu_0 N^2 A}{l}
 \quad \quad \quad \mbox{\textbf{Toroid: }}L = \frac{\mu_0 N^2 A}{2\pi r}
\end{equation}

\begin{equation}
    \label{eq2}
   \mbox{\textbf{Magnetic Potential Energy: }} U^B = \frac{1}{2}LI^2
\end{equation}

\begin{equation}
    \label{eq3}
    \mbox{\textbf{Magnetic Potential Energy Density: }}u_B = \frac{1}{2}\frac{B^2}{\mu_0}
\end{equation}

\section*{Chapter 30}

\begin{equation}
    \label{eq3}
    \mbox{\textbf{EM Waves: }}
    {E_x}(z,t) = {E_0}\sin(kx - \omega t)\hat{i} \\ \mbox{and} \\ {B_x}(z,t) = {B_0}\sin(kx - \omega t)\hat{j}
\end{equation}


\begin{equation}
    \label{eq3}
    \mbox{\textbf{Poynting Vector: }}
    \vec{S} = \frac{1}{\mu_0}\vec{E} \times \vec{B},\quad S_\mathrm{av} = \frac{1}{\mu_0}E_\mathrm{rms}B_\mathrm{rms}
\end{equation}

\begin{equation}
    \label{eq3}
    \mbox{\textbf{Electromagnetic Wave Power: }}
    P = \iint \vec{S} \cdot \vec{dA}
\end{equation}

\begin{equation}
    \label{eq3}
    \mbox{\textbf{Speed of Light: }}
    c = \; \frac{E_0}{B_0} = \; \frac{1}{\sqrt{\epsilon_0\mu_0\kappa}} = \; \frac{\omega}{k}  = \; 3.0 \cdot 10^8 \  \si{\meter\per\second}
\end{equation}

\begin{equation}
    \label{eq3}
    \mbox{\textbf{Root Mean Squared: }}
    E_\mathrm{rms}^{2} = \frac{1}{2}E_\mathrm{max}^2 \\ \mbox{and} \\ B_\mathrm{rms}^{2} = \frac{1}{2}B_\mathrm{max}^2
\end{equation}

\begin{equation}
    \label{eq3}
    \mbox{\textbf{Electromagnetic Energy Density: }}
    u_B = \frac{\epsilon_0}{E_{0}^2} = \frac{B_0^2}{\mu_0} = \sqrt{\frac{\epsilon_0}{\mu_0}}E_0 B_0
\end{equation}


\end{document}
